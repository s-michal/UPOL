\documentclass[titlepage, a4paper, 12pt]{article}
\usepackage[utf8]{inputenc}
\usepackage{fancyvrb}
\usepackage{xcolor}
\usepackage{color}
\usepackage{listings}
\usepackage{mathtools} 
\usepackage{amsmath} 
\usepackage{graphicx}



\begin{document}
\begin{titlepage}
   \vspace*{\stretch{2.0}}
   \begin{flushleft}
      \Huge\textbf{\LaTeX}\\
      \Large\textit{and what I can do in it}\\
      \vspace{10pt}
      \normalsize{Dominika Gajdová}\\
      \today\\
   \end{flushleft}
   \vspace*{\stretch{2.0}}
\end{titlepage}


\parindent=0cm

\newpage



\tableofcontents
\newpage
\section{Programming}
As a computer science major, I should be able to project any piece of code I~find necessary. 

\subsection{C language}
Here is a function for returning the maximum of two values written in good old C: 
    

\begin{lstlisting}[language = C]

int max(int a, int b){	
if (a > b)
   return a;
else if (a == b)
   return printf("equal");
else
   return b;
}
\end{lstlisting}

If you want to call the function with given parametres, you do it like this: 

\begin{lstlisting}
max(10, 20)   -> 20
max(14, 14)   -> equal
\end{lstlisting}

\subsection{Scheme}
What about a recursive procedure for calculating the value of factorial 

in~Scheme?\\


\begin{lstlisting}[language=Lisp]
(define fac
   (lambda (n)
      (if (= n 0)
         1
         (* n (fac (- n 1))))))
\end{lstlisting}

If you want to call the function with a given parametre, you do it like this: 

\begin{lstlisting}
(fac 4)  -> 24
(fac 5)  -> 120
\end{lstlisting}
\newpage

\section{Math}
Because I study computer science, I have to have some mathematical skills otherwise I would not be able to write a recursive procedure for getting the~value of a factorial.

\subsection{Analytical geometry}
How do I calculate the length of a side of a 2d object if I only know the~coordinates? That's easy: 

$$|AB|= B-A = \sqrt{(v_1-u_1)^2 + (v_2-u_2)^2}$$\\
$A$, $B$ are points of the abscissa and $v_1$, $v_2$, $u_1$, $u_1$, are vectors.\\
The area of a triangle with vertices at $(x_1, y_1) , (x_2, y_2), (x_3, y_3)$ is:
$$=\pm\frac{1}{2}\begin{bmatrix} 
x_1 & y_1 & 1 \\
x_2 & y_2 & 1 \\  
x_3 & y_3 & 1
\end{bmatrix} 
$$
$$=\pm y_2(x_1y_2 + y_1x_3 +y_3x_2 - y_2x_3 - y_1x_2 - x_1y_3)$$


\subsection{Factorial}
Everyone knows that the way to calculate a factorial of a number is to multiply all the descending numbers beginning with the wanted number. \\
For example:

$$5! = 5\cdot4\cdot3\cdot2\cdot1=120$$

So the slightly more mathematical way of saying this is:

$$n! = n(n-1)! $$

Math would not be math without another mathematical definiton, right? 
This is a recursive\footnote{A recursive procedure is a procedure that is applied within itself when defining it.} definition of a factorial:

\[
  Fac(n) =
  \begin{cases}
                                   1 & \text{if $n=0$} \\                                   
  n \cdot Fac(n -1) & \text{if $n>1$}
  \end{cases}
\]
\newpage

\section{Images}
Inserting images is quite necessarry nowadays. How would a document look with only math and coding? I'm not writing a boring textbook.

\begin{figure}[h]
\centering
\includegraphics[scale=0.26]{samsung}
\caption{A flat design of Samsung Galaxy 9}
\end{figure}

Both designs were made in Adobe Illustrator.
\begin{figure}[h]
\centering
\includegraphics[scale=0.26]{watches}
\caption{A flat design of watches}
\end{figure}


\end{document}
